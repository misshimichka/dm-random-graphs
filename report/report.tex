\documentclass{report}
\usepackage[utf8]{inputenc}
\usepackage[T1,T2A]{fontenc}
\usepackage[english,russian]{babel}
\usepackage{amsfonts}
\usepackage{amsmath}
\usepackage{upgreek}
\usepackage{amssymb}
\usepackage{wasysym}
\usepackage{graphicx}
\usepackage{amsthm}
\usepackage{hyperref}

\title{Проект по случайным графам}
\author{Чегодаева Таисия и Купряков Дмитрий, ПАДИИ, 2 курс}

\begin{document}
\maketitle

\part{Исследование свойств характеристики.}
\chapter{Исследовать, как ведет себя числовая характеристика $\tau$ в зависимости от параметров распределений $\theta$ и $\nu$, зафиксировав размер выборки и параметр процедуры построения графа.}
\section{Характеристика $\tau^{KNN}$.}
\subsection{Распределение LogNormal с $\mu$ = 0 и параметром $\theta$.}
Зафиксируем размер выборки $n = 100$ и количество соседей $k = 5$. Число итераций для метода Монте-Карло равно 1000.
\newline
\newline
Будем перебирать $\theta = \{0.001, 0.01, 0.1, 1, 2, 5, 10, 15, 20, 25, 30, 40, 50, 75, 100\}$.
\newline
\newline
Результаты показали (см. \href{knn\_lognormal\_fixed\_graph\_parameters.png}{картинку}), что усредненная характеристика $\tau^{KNN}$ при $\theta \in [1, + \infty)$ принимает значения $\in [194, + \infty)$, а на $[0, 1]$ колеблется в окрестности числа 189.

\subsection{Распределение Exp с параметром $\lambda$.}
Зафиксируем размер выборки $n = 100$ и количество соседей $k = 5$. Число итераций для метода Монте-Карло равно 1000.
\newline
\newline
Будем перебирать $\nu = \{0.0001, 0.001, 0.01, 0.1, 1, 2, 5, 10, 15, 20, 25, 30, 50, 75, 100\}$.
\newline
\newline
Результаты показали (см. \href{knn\_exp\_fixed\_graph\_parameters.png}{картинку}), что усредненная характеристика $\tau^{KNN}$ принимает значения в окрестности числа 189 независимо от параметра $\nu$.
\newline
\newline
\newline
Остальное запишу завтра, жаворонок хочет спатеньки...

\end{document}
