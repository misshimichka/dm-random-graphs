\documentclass{report}
\usepackage[utf8]{inputenc}
\usepackage[T1,T2A]{fontenc}
\usepackage[english,russian]{babel}
\usepackage{amsfonts}
\usepackage{amsmath}
\usepackage{upgreek}
\usepackage{amssymb}
\usepackage{wasysym}
\usepackage{graphicx}
\usepackage{amsthm}
\usepackage{hyperref}

\title{Проект по случайным графам}
\author{Чегодаева Таисия и Купряков Дмитрий, ПАДИИ, 2 курс}

\begin{document}
\maketitle

\part{Исследование свойств характеристики.}
\chapter{Исследовать, как ведет себя числовая характеристика $\tau$ в зависимости от параметров распределений $\theta$ и $\nu$, зафиксировав размер выборки и параметр процедуры построения графа.}

\section{Характеристика $\tau^{KNN}$.}
\subsection{Распределение SkewNormal с параметром $\alpha$.}
Зафиксируем размер выборки $n = 100$ и количество соседей $k = 5$. Число итераций для метода Монте-Карло равно 1000.
\newline
\newline
Будем перебирать $\theta = \{0.001, 0.01, 0.1, 0.5, 0.75, 1, 3, 5, 10, 15, 20, 50, 100, 500, 1000\}$.
\newline
\newline
\href{https://github.com/misshimichka/dm-random-graphs/blob/fbf3dd1bec269816312776019ca766ef0ce871b6/report/fix_construct_skewnorm_max_deg_knn.png}{Результаты}
\newline
\newline
Усредненная характеристика $\tau^{KNN}$ при любых значениях параметра $\thau$ приближенно равна 9, но при больших значениях это приближение становится более \href{https://github.com/misshimichka/dm-random-graphs/blob/dmitrii/report/report/fix_construct_skewnorm_alpha2avg_max_deg_knn.png}{заметным}. 

\subsection{Распределение Normal с параметром-дисперсией $\sigma$ и матожиданием 0.}
Зафиксируем размер выборки $n = 100$ и количество соседей $k = 5$. Число итераций для метода Монте-Карло равно 1000.
\newline
\newline
Будем перебирать $\nu = \{0.001, 0.01, 0.1, 0.5, 0.75, 1, 3, 5, 10, 15, 20, 50, 100, 500, 1000\}$.
\newline
\newline
\href{https://github.com/misshimichka/dm-random-graphs/blob/dmitrii/report/report/fix_construct_norm_max_deg_knn.png}{Результаты}
\newline
\newline
Усредненная характеристика $\tau^{KNN}$ принимает значения в окрестности числа 9 независимо от параметра $\nu$. Но при больших значениях параметра можно заметить \href{https://github.com/misshimichka/dm-random-graphs/blob/dmitrii/report/report/fix_construct_norm_sigma2avg_max_deg_knn.png}{здесь}, что характеристика $\tau^{KNN}$ начинает отклоняться от своего среднего значения.

\section{Характеристика $\tau^{dist}$.}
\subsection{Распределение SkewNormal с параметром $\alpha$.}
Зафиксируем размер выборки $n = 100$ и расстояние $dist = 1$. Число итераций для метода Монте-Карло равно 1000.
\newline
\newline
Будем перебирать \\
$\theta = \{0.001, 0.01, 0.1, 0.5, 0.75, 1, 3, 5, 10, 15, 20, 50, 100, 500, 1000, 10000, 150000, 300000, 500000, 15000000\}$.
\newline
\newline
\href{https://github.com/misshimichka/dm-random-graphs/blob/dmitrii/report/report/fix_construct_skewnorm_mis_dist.png}{Результаты}. 
\newline
\newline
Характеристика $\tau^{dist}$ при $\theta \in (0, 1)$ принимает в среднем значение 5, а при больших $\theta$ принимает значения, близкие к 3. Это хорошо видно на \href{https://github.com/misshimichka/dm-random-graphs/blob/dmitrii/report/report/fix_construct_skewnorm_alpha2avg_mis_dist.png}{графике}.

\subsection{Распределение Normal с параметром-дисперсией $\sigma$ и матожиданием 0.}
Зафиксируем размер выборки $n = 100$ и расстояние $dist = 5$. Число итераций для метода Монте-Карло равно 1000.
\newline
\newline
Будем перебирать \\
$\nu = \{0.001, 0.01, 0.1, 0.5, 0.75, 1, 3, 5, 10, 15, 20, 50, 100, 500, 1000, 10000, 150000, 300000, 500000, 15000000\}$
\newline
\newline
\href{https://github.com/misshimichka/dm-random-graphs/blob/dmitrii/report/report/fix_construct_norm_mis_dist.png}{Результаты}
\newline
\newline
Характеристика $\tau^{dist}$ при $\nu \in (0, 0.5)$ принимает значение 1 (т.е. при таких $\nu$ граф -- полный). С увеличением параметра растет среднее значение характеристики (можно посмотреть \href{https://github.com/misshimichka/dm-random-graphs/blob/dmitrii/report/report/fix_construct_norm_sigma2avg_mis_dist.png}{здесь}).

\chapter{Исследовать, как ведет себя числовая характеристика $\tau$ в зависимости от параметров процедуры построения графа и размера выборки при фиксированных значениях $\theta = \theta_0$ и $\nu = \nu_0$.}
\section{Характеристика $\tau^{KNN}$.}
\subsection{Распределение SkewNormal с параметром $\alpha_0=1$.}
Будем перебирать параметры с 1000 итерациями метода Монтэ-Карло:\\
1. $n\_samples = \{[1, 5, 10, 25, 50, 100, 300]\}$\\
2. $k\_neighbours = \{1, 3, 5, 7, 9, 15, 20\}$

\href{https://github.com/misshimichka/dm-random-graphs/blob/dmitrii/report/report/fix_alpha_skewnorm_max_deg_knn.png}{Результаты}

Можно заметить, что средняя величина характеристики $\tau^{KNN}$ увеличивается, по мере роста перебираемых параметров. Но также часто встречаются ситуация, когда среднее значение совпадает с реальным.

\subsection{Распределение Normal с параметром-дисперсией $\sigma_0=1$ и матожиданием 0.}
Будем перебирать параметры с 1000 итерациями метода Монтэ-Карло:\\
1. $n\_samples = \{[1, 5, 10, 25, 50, 100, 300]\}$\\
2. $k\_neighbours = \{1, 3, 5, 7, 9, 15, 20\}$

\href{https://github.com/misshimichka/dm-random-graphs/blob/dmitrii/report/report/fix_sigma_norm_max_deg_knn.png}{Результаты}

Можем наблюдать такую же тенденцию -- с ростом параметров растет среднее значение характеристики, даже значения принимаются такие же со сдвигом на небольшой $\epsilon$.

\section{Характеристика $\tau^{dist}$.}
\subsection{Распределение SkewNormal с параметром $\alpha_0=1$.}
Будем перебирать параметры с 1000 итерациями метода Монтэ-Карло:\\
1. $n\_samples = \{[1, 5, 10, 25, 50, 100, 300]\}$\\
2. $dists = \{0.001, 0.01, 0.1, 0.5, 1, 3, 5\}$

\href{https://github.com/misshimichka/dm-random-graphs/blob/dmitrii/report/report/fix_alpha_skewnorm_mis_dist.png}{Результаты}

Можно заметить, что больше всего на значение характеристики $\tau^{dist}$ влияет параметр $n\_samples$, а с увеличением параметра $dist$ увеличивается количество ребер из-за этого уменьшается количество независимых вершин. 

\subsection{Распределение Normal с параметром-дисперсией $\sigma_0=1$ и матожиданием 0.}
Будем перебирать параметры с 1000 итерациями метода Монтэ-Карло:\\
1. $n\_samples = \{[1, 5, 10, 25, 50, 100, 300]\}$\\
2. $k\_neighbours = \{1, 3, 5, 7, 9, 15, 20\}$

\href{https://github.com/misshimichka/dm-random-graphs/blob/dmitrii/report/report/fix_sigma_norm_mis_dist.png}{Результаты}

Для каждого значения параметра $n\_samples$ можем заметить довольно плотное распределение среднего значения характеристики $\tau^{dist}$, но с ростом этого параметра растет количество выбросов и колебания.

\end{document}
