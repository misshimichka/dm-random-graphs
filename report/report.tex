\documentclass{report}
\usepackage[utf8]{inputenc}
\usepackage[T1,T2A]{fontenc}
\usepackage[english,russian]{babel}
\usepackage{amsfonts}
\usepackage{amsmath}
\usepackage{upgreek}
\usepackage{amssymb}
\usepackage{wasysym}
\usepackage{graphicx}
\usepackage{amsthm}
\usepackage{hyperref}

\title{Проект по случайным графам}
\author{Чегодаева Таисия и Купряков Дмитрий, ПАДИИ, 2 курс}

\begin{document}
\maketitle

\part{Исследование свойств характеристики.}
\chapter{Исследовать, как ведет себя числовая характеристика $\tau$ в зависимости от параметров распределений $\theta$ и $\nu$, зафиксировав размер выборки и параметр процедуры построения графа.}
Замечание: ссылки на картинки пока что не кликабельные, но сами картинки лежат в той же папке, что и отчет.
\section{Характеристика $\tau^{KNN}$.}
\subsection{Распределение LogNormal с $\mu$ = 0 и параметром $\theta$.}
Зафиксируем размер выборки $n = 100$ и количество соседей $k = 5$. Число итераций для метода Монте-Карло равно 1000.
\newline
\newline
Будем перебирать $\theta = \{0.001, 0.01, 0.1, 1, 2, 5, 10, 15, 20, 25, 30, 40, 50, 75, 100\}$.
\newline
\newline
Результаты: \texttt{knn\_lognormal\_fixed\_graph\_parameters.png}.
\newline
\newline
Усредненная характеристика $\tau^{KNN}$ при $\theta \in [1, + \infty)$ принимает значения $\in [194, + \infty)$, а на $[0, 1]$ колеблется в окрестности числа 189.

\subsection{Распределение Exp с параметром $\lambda$.}
Зафиксируем размер выборки $n = 100$ и количество соседей $k = 5$. Число итераций для метода Монте-Карло равно 1000.
\newline
\newline
Будем перебирать $\nu = \{0.0001, 0.001, 0.01, 0.1, 1, 2, 5, 10, 15, 20, 25, 30, 50, 75, 100\}$.
\newline
\newline
Результаты: \texttt{knn\_exp\_fixed\_graph\_parameters.png}.
\newline
\newline
Усредненная характеристика $\tau^{KNN}$ принимает значения в окрестности числа 189 независимо от параметра $\nu$.

\section{Характеристика $\tau^{dist}$.}
\subsection{Распределение LogNormal с $\mu$ = 0 и параметром $\theta$.}
Зафиксируем размер выборки $n = 100$ и расстояние $dist = 5$. Число итераций для метода Монте-Карло равно 1000.
\newline
\newline
Будем перебирать $\theta = \{0.0001, 0.001, 0.01, 0.1, 1, 2, 5, 10, 15, 20, 25, 30, 50, 75, 100, 150, 200, 250, 500, 1000\}$.
\newline
\newline
Результаты: \texttt{dist\_lognormal\_fixed\_graph\_parameters.png}. 
\newline
\newline
Характеристика $\tau^{dist}$ при $\theta \in (0, 1)$ принимает значение 50 (т.е. при таких $\theta$ граф -- полный). 
\newline
\newline
При $\theta \in [1, +\infty)$ с увеличением $\theta$ среднее значение характеристики $\tau^{dist}$ колеблется в окрестности числа 25, а сама характеристика в большинстве случаев колеблется между значениями 15 и 35.
\newline
\newline
Дополнительно смотрела на большие $\theta \in [15, 500000]$, начиная с некоторого момента нижняя граница колебаний $\tau^{dist}$ выравнивается (как раз где-то до $\tau^{dist} = 25$), соотвественно, среднее значение немного увеличивается и колеблется около 27.
\newline
\newline
Результаты для больших $\theta$: \texttt{dist\_lognormal\_big\_theta\_fixed\_graph\_parameters.png}

\subsection{Распределение Exp с параметром $\lambda$.}
Зафиксируем размер выборки $n = 100$ и расстояние $dist = 5$. Число итераций для метода Монте-Карло равно 1000.
\newline
\newline
Будем перебирать $\nu = \{0.0001, 0.001, 0.01, 0.1, 1, 2, 5, 10, 15, 20, 25, 30, 50, 75, 100, 150, 200, 250, 500, 1000\}$.
\newline
\newline
Результаты: \texttt{dist\_exp\_fixed\_graph\_parameters.png}.
\newline
\newline
Характеристика $\tau^{dist}$ при $\nu \in (0, 1)$ принимает значение 50 (т.е. при таких $\nu$ граф -- полный). 
\newline
\newline
При больших $\nu$ среднее значение $\tau^{dist}$ стремится к 1. 
\newline
\newline
Дополнительно смотрела на большие $\nu \in [15, 500000]$, на отдельной картинке отлично видно это стремление к 1.
\newline
\newline
Результаты для больших $\nu$: \texttt{dist\_lexp\_big\_nu\_fixed\_graph\_parameters.png}

\chapter{Исследовать, как ведет себя числовая характеристика $\tau$ в зависимости от параметров процедуры построения графа и размера выборки при фиксированных значениях $\theta = \theta_0$ и $\nu = \nu_0$.}
\section{Характеристика $\tau^{KNN}$.}
\subsection{Распределение LogNormal с $\mu$ = 0 и $\theta = \theta_0 = 1$.}

\subsection{Распределение Exp с параметром $\nu = \nu_0 = \frac{1}{\sqrt{e^2 - e}}$.}

\section{Характеристика $\tau^{dist}$.}
\subsection{Распределение LogNormal с $\mu$ = 0 и $\theta = \theta_0 = 1$.}

\subsection{Распределение Exp с параметром $\nu = \nu_0 = \frac{1}{\sqrt{e^2 - e}}$.}

\end{document}
